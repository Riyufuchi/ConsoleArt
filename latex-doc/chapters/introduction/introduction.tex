%!TeX root =  ../../ConsoleArt.tex

This application originally began as a purely CLI-based project (which is also the reason for its name, \textit{ConsoleArt}). As the project evolved, a \gls{gui} was later added. Development started at a time when C++17 was the newest standard, so the use of modern features such as smart pointers was adopted gradually throughout the project.

\section{Technologies Used}

\subsection{SDL2}
This project is primarily written in C++ and uses \gls{sdl2} for window management, rendering, and input handling. Since SDL2 does not provide native GUI widgets, the application uses a custom library to implement UI components on top of SDL.

\subsection{STB}
The project utilizes the single-header \texttt{stb} libraries, primarily \texttt{stb\_image}, to load image formats such as PNG, JPEG, and HDR files.

\subsection{TinyFileDialogs}
\texttt{TinyFileDialogs} is used to provide simple, cross-platform file selection dialogs, allowing users to select images without relying on large external GUI toolkits.

\subsection{ConsoleLib}
\texttt{ConsoleLib} is my custom library developed to simplify the creation of CLI applications. 
It provides an abstract \texttt{IConsole} interface and several platform-specific implementations:

\begin{itemize}
    \item \textbf{DefaultConsole} – A basic implementation using standard output (\texttt{std::cout}) without any additional formatting features.
    \item \textbf{UnixConsole} – Adds support for text coloring and formatting through ANSI escape codes commonly available on Unix-like systems.
    \item \textbf{WindowsConsole} – Inherits from \texttt{UnixConsole} and enables UTF-8 output on Windows while ensuring compatibility with ANSI escape codes.
\end{itemize}

The library also includes additional utility modules such as an argument parser and other helpers commonly required in CLI tools, making it reusable across multiple projects.

\subsection{ComponentsSDL}
\texttt{ComponentsSDL} is my lightweight UI component framework built on top of \gls{sdl2}. 
Since SDL2 does not provide native GUI widgets, this library introduces a structured way to 
create interface elements within an SDL application.

The framework provides reusable components such as panels, buttons, and layout helpers, 
allowing the application to assemble its interface in a modular and maintainable manner. 
Each component is responsible for its own rendering and event handling, making the system 
extensible and easy to integrate into SDL-based projects.

\section{Commiting a git commit}
\begin{table} [H]
	\centering
	\catcode`\-=12
	\begin{tabular}[c]{|| c | c ||}
	\hline
		\multicolumn{2}{||c||}{Commit type cheat table} \\
	\hline
 		 \textbf{Type} & \textbf{Repository change} \\
	\hline
		\textcolor{green}{Feat:} & A new feature is added \\
	\hline
		\textcolor{blue}{Fixed:} & A bug is fixed \\
	\hline
		\textcolor{orange}{Docs:} & A documentation is updated \\
	\hline
		\textcolor{black}{Refactor:} & A code change that is not affecting functionality \\
	\hline
		\textcolor{brown}{Test:} & A code change in unit tests \\
	\hline
		\textcolor{gray}{Chore:} & A repository maintenance action \\
	\hline
		\textcolor{magenta}{Version x.y} & When new version is released for clearer commit history \\
	\hline
	\end{tabular}
	\caption{Types of commit headers}
	\label{table:cmtHeaders}
\end{table}